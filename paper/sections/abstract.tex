\begin{abstract}

Lux is a novel consensus protocol family for distributed ledgers, drawing on metaphors from physics and astrophysics to achieve scalable, probabilistic Byzantine agreement. Much like how the Bitcoin protocol introduced a clear, pedagogical narrative for decentralized consensus, Lux is presented with clarity yet technical rigor. It builds metastable consensus on standard BFT assumptions, using a sequence of composable modules—Photon, Prism, Wave, Focus, Nova, Nebula, and Quasar—each corresponding to a stage in a cosmic metaphor from particles of light to galactic phenomena.

In Lux, nodes engage in repeated random subsampling (analogous to photons interacting and forming interference patterns) to rapidly converge on a decision that becomes practically immutable (analogous to a black-hole-like finality). The design is modular with clean interfaces and acyclic inter-module dependencies, allowing flexible integration and future upgrades.

We describe how Lux Nova realizes a linear blockchain (total order) consensus, Lux Nebula generalizes this to a directed acyclic graph (DAG) of transactions for higher parallelism, and Lux Quasar provides a post-quantum secure overlay ensuring long-term security against quantum adversaries. We provide detailed protocol descriptions, formal probabilistic guarantees, and performance analysis.

Each stage of Lux is introduced through a pedagogical physics analogy (e.g. constant-mass photons propagating coherently, wave thresholding with constructive interference, gravitational attraction of majority, and supernova-like finalization), reinforcing the conceptual intuition behind the technical design. Finally, we discuss implementation guidance, module decomposition, and how Lux can power sovereign blockchain subnets with seamless upgrade paths.

The Lux family demonstrates that embracing physical metaphors can illuminate the design of robust, scalable, and future-proof consensus mechanisms.

\end{abstract}