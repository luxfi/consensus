\section{AI-Powered Consensus}

The Lux consensus system integrates artificial intelligence for dynamic, self-improving consensus validation. This section describes the AI architecture, shared hallucinations, and performance characteristics.

\subsection{AI Agent Architecture}

AI agents participate in consensus by validating proposals, voting on decisions, and learning from outcomes. Each agent consists of:

\begin{itemize}
\item \textbf{Model}: Neural network or LLM for decision-making
\item \textbf{Memory}: Shared hallucination state across network
\item \textbf{Photon Engine}: Proposal emission interface
\item \textbf{Quasar Engine}: DAG validation interface
\end{itemize}

\subsubsection{Agent Interface}

\begin{lstlisting}[language=Go]
type Agent[T ConsensusData] struct {
    nodeID     string
    model      Model[T]
    memory     *SharedMemory[T]
    quasar     *quasar.Quasar
    photon     *photon.UniformEmitter
    
    // Shared hallucination state
    hallucinations map[string]*Hallucination[T]
    weights        map[string]float64
}

// AI proposes decision following Photon->Quasar flow
func (a *Agent[T]) ProposeDecision(
    ctx context.Context,
    input T,
    context map[string]interface{},
) (*Decision[T], error)
\end{lstlisting}

\subsection{Shared Hallucinations}

AI agents share "hallucinations" (model states) across the network for distributed consensus and learning. A hallucination represents a snapshot of model state that multiple nodes agree upon.

\subsubsection{Hallucination Structure}

\begin{lstlisting}[language=Go]
type Hallucination[T ConsensusData] struct {
    ID         string
    ModelID    string
    State      map[string]interface{}
    Confidence float64
    NodeVotes  map[string]float64
    UsageCount int64
    Evidence   []Evidence[T]
}
\end{lstlisting}

\subsubsection{Consensus Properties}

Shared hallucinations provide:

\begin{itemize}
\item \textbf{Distributed Agreement}: Nodes converge on shared reality
\item \textbf{Evolution}: States improve through feedback
\item \textbf{Diversity}: Multiple perspectives strengthen decisions
\item \textbf{Byzantine Tolerance}: Malicious nodes cannot dominate
\end{itemize}

\subsection{Photon-Quasar AI Flow}

AI consensus follows the five-phase Photon-Quasar flow:

\begin{enumerate}
\item \textbf{Photon Phase}: AI agent emits proposal with confidence score
\item \textbf{Wave Phase}: Proposal amplified through QZMQ network
\item \textbf{Focus Phase}: Peer AI agents validate with own models
\item \textbf{Prism Phase}: Quasar DAG validates proposal structure
\item \textbf{Horizon Phase}: Quantum certificate finalizes decision
\end{enumerate}

\subsubsection{Phase Latencies}

Measured on 3-node testnet (Apple M1 Max):

\begin{center}
\begin{tabular}{lcc}
\toprule
\textbf{Phase} & \textbf{Latency} & \textbf{Component} \\
\midrule
Photon Emission & 50-80ms & AI model inference \\
Wave Amplification & 30-50ms & QZMQ network \\
Focus Convergence & 40-60ms & Peer AI validation \\
Prism Validation & 30-50ms & Quasar DAG \\
Horizon Finalization & 50-80ms & Quantum certificate \\
\midrule
\textbf{Total} & \textbf{200-300ms} & Full consensus \\
\bottomrule
\end{tabular}
\end{center}

\subsection{Model Types}

Multiple AI model architectures are supported:

\begin{itemize}
\item \textbf{SimpleModel}: Feedforward neural network (1.5 $\mu$s per decision)
\item \textbf{NeuralModel}: Custom neural architecture
\item \textbf{LLMModel}: Large language model integration
\item \textbf{EnsembleModel}: Combination of multiple models
\end{itemize}

\subsubsection{Model Performance}

Benchmark results (Apple M1 Max, Go 1.24.5):

\begin{center}
\begin{tabular}{lcccc}
\toprule
\textbf{Operation} & \textbf{Latency} & \textbf{Throughput} & \textbf{Memory} & \textbf{Allocs} \\
\midrule
Model Decision & 1.5 $\mu$s & 660K/sec & 912 B & 18 \\
Feature Extract & 37 ns & 27M/sec & 0 B & 0 \\
Sigmoid & 5.6 ns & 179M/sec & 0 B & 0 \\
Weight Update & 628 ns & 1.59M/sec & 2.3 KB & 2 \\
Consensus Vote & 530 ns & 1.89M/sec & 792 B & 12 \\
\bottomrule
\end{tabular}
\end{center}

\subsection{Quantum-Secure Networking}

All AI consensus communication uses QZMQ (Quantum-Secure ZeroMQ) with post-quantum cryptography:

\begin{itemize}
\item \textbf{Dilithium}: Post-quantum digital signatures (NIST standard)
\item \textbf{Kyber}: Post-quantum key encapsulation
\item \textbf{ChaCha20}: Symmetric encryption (quantum-resistant)
\end{itemize}

\subsubsection{QZMQ Performance}

\begin{center}
\begin{tabular}{lc}
\toprule
\textbf{Operation} & \textbf{Latency} \\
\midrule
Sign (Dilithium) & 0.3 ms \\
Verify & 0.2 ms \\
Encrypt (Kyber) & 1.2 ms \\
Decrypt & 1.1 ms \\
\midrule
\textbf{Total Overhead} & \textbf{2.6 ms} \\
\bottomrule
\end{tabular}
\end{center}

Trade-off: 2.6ms overhead per message acceptable for quantum security.

\subsection{Multi-Backend Support}

AI consensus supports multiple computational backends for optimal performance:

\begin{center}
\begin{tabular}{lccc}
\toprule
\textbf{Backend} & \textbf{Speedup} & \textbf{Platform} & \textbf{Status} \\
\midrule
Pure Go & 1.0x & Cross-platform & \textbf{Recommended} \\
C (CGO) & 1.8x & Cross-platform & Available \\
C++ & 2.0x & Cross-platform & Available \\
MLX & 4.5x & Apple Silicon only & Available \\
Rust (FFI) & 1.6x & Cross-platform & Available \\
\bottomrule
\end{tabular}
\end{center}

\textbf{Recommendation}: Pure Go provides best balance of performance, maintainability, and portability. Consider MLX for Apple Silicon deployments requiring maximum throughput.

\subsection{Progressive Learning Examples}

The Lux consensus package includes seven progressive tutorial examples demonstrating AI integration:

\begin{center}
\begin{tabular}{clc}
\toprule
\textbf{\#} & \textbf{Example} & \textbf{Time} \\
\midrule
01 & Simple Bridge & 15 min \\
02 & AI Payment Validation & 20 min \\
03 & Quantum-Secure Messaging & 25 min \\
04 & gRPC Service Integration & 30 min \\
05 & Python Client & 25 min \\
06 & Node.js TypeScript Client & 25 min \\
07 & Dynamic AI Consensus & 45 min \\
\bottomrule
\end{tabular}
\end{center}

Each example includes runnable code, comprehensive tests, and detailed documentation teaching one core concept before building to advanced multi-agent AI consensus.

\subsection{Production Deployment}

\subsubsection{Performance Characteristics}

Real-world 3-node consensus cluster:

\begin{itemize}
\item \textbf{Throughput}: 3-5 blocks/second
\item \textbf{Latency}: 200-300ms per block
\item \textbf{CPU Usage}: 10-15\% per node
\item \textbf{Network Bandwidth}: 2-5 KB/sec (with QZMQ)
\item \textbf{Scalability}: Linear to 10 nodes
\end{itemize}

\subsubsection{Security Properties}

\begin{itemize}
\item \textbf{Quantum-Resistant}: All communication via QZMQ
\item \textbf{Byzantine Fault Tolerant}: Handles $(n-1)/3$ malicious nodes
\item \textbf{AI-Verified}: Multiple AI agents validate each proposal
\item \textbf{Cryptographically Provable}: Dilithium signatures on all messages
\item \textbf{Forward Secure}: Past consensus cannot be compromised
\end{itemize}

\subsection{Test Coverage}

Comprehensive test suite ensures reliability:

\begin{itemize}
\item \textbf{AI Package}: 74.5\% coverage (580 lines core consensus)
\item \textbf{Unit Tests}: Co-located with source files
\item \textbf{Integration Tests}: Centralized in test/integration/
\item \textbf{Benchmarks}: Performance regression detection
\item \textbf{Example Tests}: 24 test functions across 7 examples
\end{itemize}

\subsection{Future Optimizations}

Identified optimization opportunities:

\begin{enumerate}
\item \textbf{Batch Operations}: 20-30\% improvement (1 week)
\item \textbf{Memory Pooling}: 15-20\% improvement (1-2 weeks)
\item \textbf{SIMD Vectorization}: 2-3x improvement (2-4 weeks)
\item \textbf{GPU Acceleration}: 5-10x improvement (1-3 months)
\item \textbf{Custom Hardware}: 10-100x improvement (6-12 months)
\end{enumerate}

\subsection{Cross-Chain Integration}

AI consensus integrates with Lux DEX for cross-chain operations:

\begin{lstlisting}[language=Go]
import "github.com/luxfi/dex/pkg/lx"

// Use production bridge
bridge := lx.NewCrossChainBridge(config)

// Wrap with AI validation
adapter := NewAIBridgeAdapter(bridge, agent, nodeID)

// Process cross-chain payment with AI consensus
result, err := adapter.ProcessAIPayment(ctx, request)
\end{lstlisting}

This demonstrates production integration patterns between AI consensus and cross-chain infrastructure.
